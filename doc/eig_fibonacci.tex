\documentclass{article}
\usepackage[pagebackref,letterpaper=true,colorlinks=true,pdfpagemode=none,urlcolor=blue,linkcolor=blue,citecolor=blue,pdfstartview=FitH]{hyperref}

\usepackage{amsmath,amsfonts}
\usepackage{graphicx}
\usepackage{color}


\setlength{\oddsidemargin}{0pt}
\setlength{\evensidemargin}{0pt}
\setlength{\textwidth}{6.0in}
\setlength{\topmargin}{0in}
\setlength{\textheight}{8.5in}


\setlength{\parindent}{0in}
\setlength{\parskip}{5px}

%\input{macrosblog}

%%%%%%%%% For wordpress conversion

\def\more{}

\newif\ifblog
\newif\iftex
\blogfalse
\textrue


\usepackage{ulem}
\def\em{\it}
\def\emph#1{\textit{#1}}

\def\image#1#2#3{\begin{center}\includegraphics[#1pt]{#3}\end{center}}

\let\hrefnosnap=\href

\newenvironment{btabular}[1]{\begin{tabular} {#1}}{\end{tabular}}

\newenvironment{red}{\color{red}}{}
\newenvironment{green}{\color{green}}{}
\newenvironment{blue}{\color{blue}}{}

%%%%%%%%% Typesetting shortcuts

\def\B{\{0,1\}}
\def\xor{\oplus}

\def\P{{\mathbb P}}
\def\E{{\mathbb E}}
\def\var{{\bf Var}}

\def\N{{\mathbb N}}
\def\Z{{\mathbb Z}}
\def\R{{\mathbb R}}
\def\C{{\mathbb C}}
\def\Q{{\mathbb Q}}
\def\eps{{\epsilon}}

\def\bz{{\bf z}}

\def\true{{\tt true}}
\def\false{{\tt false}}

%%%%%%%%% Theorems and proofs

\newtheorem{exercise}{Exercise}
\newtheorem{theorem}{Theorem}
\newtheorem{lemma}[theorem]{Lemma}
\newtheorem{definition}[theorem]{Definition}
\newtheorem{corollary}[theorem]{Corollary}
\newtheorem{proposition}[theorem]{Proposition}
\newtheorem{example}{Example}
\newtheorem{remark}[theorem]{Remark}
\newenvironment{proof}{\noindent {\sc Proof:}}{$\Box$} %\medskip} 
%%%%%%%%% I added
\newtheorem{assumption}{Assumption}
%%%%%%%%

\begin{document}

\section{Abstract}
\begin{itemize}
 \item sec 5.6
 \item estimation of Fibonacci numbers via golden mean
 \item eigenvalue and eigenvector and matrix diagonalization
\end{itemize}

\section{Problem}
Fibonacci numbers are defined recursively by
$$F_{0} = 0, F_{1} = 1, F_{n+1} = F_{n} + F_{n-1}, \forall n\ge 2.$$
What is $F_{100}$?
Our answer to this problem is surprisingly simple but a big number:
$$F_{100} = [\lambda_{1}^{100}/\sqrt 5] \approx 3.54\cdot 10^{20}$$
where $[a]$ means integer part of $a$ and $\lambda_{1}$ is the golden mean:
$$\lambda_{1} = \frac{1+ \sqrt 5}{2}.$$

\section{Analysis}
We write 
$$\left[ 
\begin{array}
 {ll}
 F_{n} \\
 F_{n+1}
\end{array}
\right]
= 
\left[ 
\begin{array}
 {ll}
 0 & 1 \\
1 & 1
\end{array}
\right]
\left[ 
\begin{array}
 {ll}
 F_{n-1} \\
 F_{n}
\end{array}
\right]
$$
Then, eigenvectors of $A$ are given by
$$\lambda_{1} = \frac{1+\sqrt 5}{2}, \ v_{1} = [1, \lambda_{1}]'$$
and 
$$\lambda_{2} = \frac{1 - \sqrt 5}{2}, \ v_{2} = [1, \lambda_{2}]'.$$
If we denote
$$V = [v_{1}, v_{2}], \ \Lambda = diag(\lambda_{1}, \lambda_{2}),$$
then we have diagonalization of $A$ as
$$AV = V\Lambda, \hbox{ or } A = V \Lambda V^{-1}.$$

Therefore, 
$$
\left[
\begin{array}
 {ll}
 F_{n} \\
 F_{n+1}
\end{array}
\right]
= 
V \Lambda^{n} V^{-1} 
\left[
\begin{array}
 {ll}
 F_{0} \\
 F_{1}
\end{array}
\right].
$$
So we can compute 
$$F_{100} = (\lambda_{1}^{100} - \lambda_{2}^{100})/(\lambda_{1} - 
\lambda_{2}).$$


\end{document}
